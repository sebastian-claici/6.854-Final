
%%% Local Variables:
%%% mode: latex
%%% TeX-master: t
%%% End:

\documentclass[12pt]{article}

\usepackage{fullpage}

% Maths
\usepackage{amsmath,amssymb,amsthm}

% Algorithms
\usepackage{algpseudocode,algorithm}

\title{From Boston to San Francisco:\\ A Survey of Shortest Paths Algorithms in Planar Graphs}
\author{Stuart Baker, \"{O}mer Cerraho\u{g}lu, Sebastian Claici}
\date{\today}

\begin{document}
\maketitle

\begin{abstract}

\end{abstract}

\section{Introduction}
\label{sec:introduction}

\section{Background}
\label{sec:background}


\section{Single source shortest paths}
\label{sec:single-source-short}

We call an edge $uv$ relaxed if $d(v) \leq d(u) + c(u,v)$. We call the assignment
\[
d(v) \gets \min\{d(v), d(u)+c(u,v)\}
\]
the relaxation of vertex $v$. We know that the labels give a correct shortest path distances if the shortest-path conditions are satisfied:
\begin{itemize}
  \item $d(s) = 0$,
  \item every label $d(v)$ is an upper bound on the $s-v$ distance,
  \item every edge is relaxed.
\end{itemize}

\subsection{Nonnegative edge weights}
\label{sec:nonn-edge-weights}

For a planar graph with nonnegative edge weights, Dijkstra's algorithm runs in $O(n \log n)$ as $m \leq 3n - 6$. It is possible to improve this to $O(n)$. To get there, recall that and $r$-division of a planar graph is a partition of the graph into $\Theta(n/r)$ regions of size $O(r)$ with boundary size $O(\sqrt{r})$. An $r$-division of a planar graph can be computed in linear time.

A simple $O(n\sqrt{\log n \log \log n})$ emerges quite beautifully just from the $r$-division if we set $r = \frac{\log n}{\log \log n}$. The algorithm follows a divide-and-conquer approach in which each region is processed first, followed by a clean-up phase where the results are merged.

\begin{algorithm}[!htb]
  \caption{Shortest paths in each region $R$}
  \label{alg:sssp-region}
  \begin{algorithmic}
    \ForAll {Regions $R$}
      \ForAll {Boundary nodes $v \in R$}
        \State Compute SSSP from $v$ in $R$
        \State Store $(u,v)$ distances for any two boundary nodes $u$, $v$
      \EndFor
    \EndFor
  \end{algorithmic}
\end{algorithm}

The first step is to compute the single-source shortest paths for each boundary node in each region $R$ (algorithm~\ref{alg:sssp-region}). We can now replace each region $R$ by a complete graph on $R$'s boundary nodes with shortest paths distances between any two nodes. Call this auxiliary graph $G'$. The second phase of the algorithm is to compute the SSSP from $s$ in $G'$. This gives the true shortest paths from $s$ to all the boundary nodes. Finally, we must tidy up by finding the distances from $s$ to the nodes inside each region (algorithm~\ref{alg:sssp-full}).

\begin{algorithm}[!htb]
  \caption{Clean up: shortest paths from $s$ to inside of each region $R$}
  \label{alg:sssp-full}
  \begin{algorithmic}
    \ForAll {Regions $R$}
      \ForAll {Boundary nodes $v \in R$}
        \State Set $d(v) = d_{G'}(s,v)$
        \State Compute SSSP from $v$ in $R$
      \EndFor
    \EndFor
  \end{algorithmic}
\end{algorithm}

To analyze the algorithm, we will need a few pieces of information:
\begin{itemize}
\item Total number of boundary nodes is $O(\sqrt{r})O(n/r) = O(n/\sqrt{r})$.
\item Number of nodes in $G'$ is $O(n/r)O(\sqrt{r})=O(n/\sqrt{r})$.
\item Number of edges in $G'$ is $O(n/r)O(r) = O(n)$.
\end{itemize}

Using $r=\frac{\log n}{\log \log n}$, the first phase is bounded above by
\[
  O(n\frac{\sqrt{\log \log n}}{\sqrt{\log n}} \log n) = O(n \sqrt{\log n \log \log n}),
\]
the second phase is an SSSP in a size $O(n \frac{\sqrt{\log \log n}}{\sqrt{\log n}})$ graph, and thus also $O(n \sqrt{\log n \log \log n})$, while the tidying up is a series of SSSPs in each of the regions, and has the same bound as the first phase---$O(n \sqrt{\log n \log \log n})$. The total time bound ends up $O(n \sqrt{\log n \log \log n})$.

\subsection{Arbitrary edge weights}
\label{sec:arbitr-edge-weights}

\section{Multiple source shortest paths}
\label{sec:mult-source-short}

\section{Extensions to higher genus}
\label{sec:extens-high-genus}


\bibliographystyle{plain}
\bibliography{proposal}
\end{document}